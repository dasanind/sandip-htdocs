\documentclass[conference]{IEEEtran}
\usepackage{epsfig,amsmath,amssymb}
\usepackage{acl2}
\newcommand{\proc}[1]{\ifmmode\mbox{\textsc{#1}}\else\textsc{#1}\fi}
\newcommand{\func}[1]{\ensuremath{\mbox{{\sf {\em #1}}}}}
\newcommand{\id}[1]{\ifmmode\mathit{#1}\else\textit{#1}\fi}
\newcommand{\naturals}{{\mathrm{I}\!\mathrm{N}}}
\renewcommand{\cases}[1]{\ensuremath{\left\{\begin{array}{ll}#1\end{array}\right.}}

\begin{document}

\title{Mechanized Information Flow Analysis through Inductive Assertions}

\author{
\authorblockN{Warren A. Hunt, Jr., Robert Bellarmine Krug, Sandip Ray, and William D. Young} 
\authorblockA{
Department of Computer Sciences \\
University of Texas at Austin \\
{\sf \{hunt,rkrug,sandip,byoung\} @cs.utexas.edu}}
}


% From http://es.fbk.eu/events/fmcad08/cfp.html:
% ... short papers (4 pages), describing applications, case studies, industrial
% experience reports, emerging results, or implemented tools with novel features.

\maketitle

%% Abstract
\begin{abstract}

We present a method for verifying information flow properties of
software programs using inductive assertions and theorem proving.
Given a program annotated with information flow assertions at
cutpoints, the method uses a theorem prover and operational semantics
to generate and discharge verification conditions.  This obviates the
need to develop a verification condition generator (VCG) or a
customized logic for information flow properties.  The method is
compositional: a subroutine needs to be analyzed once, rather than at
each call site.  The method is being mechanized in the ACL2 theorem
prover, and we discuss initial results demonstrating its
applicability.

\end{abstract}

% Sections

\section{Introduction}
\label{sec:intro}

Security of many critical computing systems depends on {\em
information flow policies} that prohibit access to sensitive
information without proper authorization.  With the increasing
application of software systems to secure applications, it is vital
to ensure that a software implementation properly enforces information
flow policies.  The goal of this paper is to develop techniques for
mechanized information flow analysis.

In its simplest form, modeling an information flow policy involves
labeling certain program variables as classified (or high security),
with the requirement that the value of an unclassified variable is not
influenced by the initial values of any classified variable.  Such a
policy can be formalized by {\em noninterference}~\cite{goguen82}.  A
deterministic program satisfies the policy if, from a pair of initial
states differing only in classified variables, any pair of
computations leads to final states with identical values for
unclassified variables.  Noninterference naturally generalizes to a
lattice of security levels.

This paper proposes a method for verifying information flow properties
of software programs through general-purpose theorem proving.
Programs are formalized through an {\em operational semantics} of the
underlying language defined by an interpreter that specifies the
effect of executing instructions on the system state.  Our approach
uses {\em inductive assertions}.  Given a program annotated with
assertions at cutpoints, we derive {\em verification conditions} that
ensure requisite information flow control, to be discharged with a
theorem prover.

A key feature of our approach is that it obviates the need for
implementing a custom verification condition generator (VCG) for
information flow properties of the underlying language constructs.
Instead, we show how to configure an off-the-shelf theorem prover to
{\em mimic} a VCG through symbolic simulation of the operational
model.  The method is inspired by, and an extension of, our previous
work~\cite{symbolic} which showed how to prove functional correctness
via symbolic simulation.  The method is compositional; properties of
subroutines can be verified individually rather than at each call
site.  We demonstrate the method by analyzing a small but illustrative
program with the ACL2 theorem prover.

\section{Basic Framework}
\label{sec:basic-framework}

We use operational semantics to model a
program by its effects on the machine states.  A
state is a tuple of values of all machine variables---the program
counter (pc), registers, memory, etc.  The semantics is then given by
a transition function $\func{next}:S\rightarrow S$ where $S$ is the
set of states: for a state $s$, $\func{next}\,(s)$ returns the state
after executing one instruction from $s$.  Executions are modeled by
the function $\func{run}:S\times\naturals\rightarrow S$ which returns
the state after $n$ transitions from $s$.

\[
\func{run}\,(s,n)
\triangleq
\cases{
        s                              & \mbox{if $n=0$} \\
        \func{run}\,(\func{next}\,(s),n-1) & \mbox{otherwise} 
      }
\]

To illustrate how to formally specify information flow properties of
programs using operational semantics, we consider the simple version
of noninterference mentioned in Section~\ref{sec:intro}.  Assume a
partition of the variables into sets $H$ (high) and $L$ (low),
corresponding to classified and unclassified data.  Furthermore,
assume that we have two predicates \func{poise} and \func{exit} on
set $S$.  For any state $s$, $\func{poise}\,(s)$ stipulates that $s$
is poised to initiate execution of the program of interest: it
specifies that the program is in the current call frame and the pc
points to its first instruction.  The predicate \func{exit}
characterizes the termination states.  To formalize the
noninterference statement, we make use of the function \func{esteps}
below, which returns, for any state $s$, the number of transitions to
the first \func{exit} state reachable from $s$ (if such an \func{exit}
state exists).
\[
\begin{array}{l}
\func{estpt}\,(s,i)
\triangleq
\cases{
        i                              & \mbox{if $\func{exit}\,(s)$} \\
        \func{estpt}\,(\func{next}\,(s), i+1) & \mbox{otherwise} 
      }
\\
\func{esteps}(s) \triangleq\func{estpt}(s,0)
\end{array}
\]

\noindent
The definition of \func{estpt} is {\em partial}: its return value is
unspecified if no \func{exit} state is reachable from $s$.  Defining a
recursive function generally requires a termination proof.  However,
since the definition is tail-recursive, it is admissible in theorem
provers whose logics support Hilbert's choice
operator~\cite{jar:defpun}.

A formalization of noninterference is shown in
Fig.~\ref{fig:noninterference},
\begin{figure}
\begin{center}
\fbox{
\begin{minipage}{3.3in}
\begin{math}
\begin{array}{l}
\func{pre}\,(s,s') \triangleq \func{poise}\,(s)\wedge \func{poise}\,(s')\wedge (\bigwedge_{l\in L} l(s)=l(s')) \\
\func{post}\,(s,s')  \triangleq  (\bigwedge_{l\in L} l(s)=l(s')) \\
\func{nexte}(s) \triangleq \func{run}(s,\func{esteps}(s))
\end{array}
\end{math}

\vspace*{4pt}

\mbox{{\bf Noninterference Condition:}}  

$\hspace*{.3cm} \func{pre}\,(s,s')\wedge\func{exit}\,(\func{run}\,(s,n)) \\
 \Rightarrow\func{exit}\,(\func{nexte}(s')) \wedge
                \func{post}\,(\func{nexte}\,(s),\func{nexte}\,(s'))$
\end{minipage}
}
\end{center}
\caption{Formal Definition of Noninterference.  Here $l(s)$ is assumed
to be the value of variable $l$ in state $s$.}
\label{fig:noninterference}
\end{figure}
and can be paraphrased as follows.  ``Let $s$ and $s'$ be any two
states poised to execute the program, such that the variables in $L$
have the same valuation in both $s$ and $s'$.  Suppose that there is
an \func{exit} state reachable from $s$.  Then the following
conditions hold.  (1)~There is an \func{exit} state reachable from
$s'$.  (2) Let $s_0$ and $s_0'$ be the first \func{exit} states
reachable from $s$ and $s'$ respectively; then $s_0$ and $s_0'$ have
the same valuation for all variables in $L$.'' 

Note that the statement of a practical information flow property might
differ from the above.  For instance, one might have a lattice of security
levels.  Or one might be interested only in a subset $L'\subseteq L$
at \func{exit}; then the conjunction in the definition of
$\func{post}$ would range over $L'$.  Nevertheless, such concerns
affect only the concrete definitions of \func{pre} and \func{post};
once they have been defined for the desired information flow
requirements, the noninterference statement can be used as is.

Our approach is based on {\em inductive assertions} which involve
annotating a program with assertions at cutpoints that include loop
tests, program entry, and exit.  For our purpose, the set of cutpoints
is characterized by a predicate \func{cut} on $S$, commonly depending
only on the pc.  One then proves that whenever the program control
reaches a cutpoint, the corresponding assertion holds.  For functional
correctness, this is achieved by a VCG as follows.  A VCG crawls over
an annotated program, generating verification conditions to be
discharged by theorem proving.  The guarantee from the VCG process is
informally stated as follows.  ``Let $p$ be any non-exit cutpoint
satisfying the assertions.  Let $q$ be the next subsequent cutpoint.
Then the assertions hold at $q$.''  It follows that the corresponding
assertion holds whenever the program control reaches a cutpoint.

How do we extend the above for information flow properties?  Since
information flow is characterized by {\em pairs of states}, the
assertions involved are formalized by a predicate $\func{assert}$ over
$S\times S$.  The associated VCG guarantee is as follows.  ``Let $p$
and $p'$ be any two corresponding non-exit cutpoints of the program
such that $\func{assert}\,(p,p')$ holds.  (See below for an
explanation of the role of ``corresponding cutpoints.'')  Let $q$ and
$q'$ be the next cutpoints from $p$ and $p'$ respectively.  Then
$\func{assert}\,(q,q')$ holds.''  Then, if additionally,
(1)~\func{assert} holds for the initial pair of states, and (2)~for
the pair of \func{exit} states \func{assert} implies \func{post}, it
follows that the pair of \func{exit} states reachable from any pair of
initial states satisfies \func{post}.

The formal verification conditions for information flow are shown in
Fig.~\ref{fig:vcg}.
\begin{figure}
\fbox{
\begin{minipage}{3.3in}
\begin{math}
\begin{array}{l}
\func{cstpt}\,(s,i)
\triangleq
\cases{i                             & \mbox{if $\func{cut}\,(s)$} \\
        \func{nextc}\,(\func{next}\,(s,i+1)) & \mbox{otherwise} 
      } \\
\func{cstps}(s) \triangleq
\cases{\func{cstpt}\,(s,0)                          & \mbox{if $\func{cstpst}\,(s,0)\in\naturals$} \\
        \omega & \mbox{otherwise} 
      } \\
\func{cut}\,({\sf D})\Leftrightarrow(\forall s: \func{cut}\,(s)) \\
\func{nextc}(s)
\triangleq\cases{\func{run}(s,\func{cstps}\,(s)) & \mbox{if $\func{cstps}\,(s)\in\naturals$} \\
        {\sf D} & \mbox{otherwise} 
      }
\end{array}
\end{math}


\vspace*{4pt}

\mbox{{\bf Verification Conditions}}  

\vspace*{1pt}

\begin{math}
\begin{array}{ll}

1. & \func{C}\,(s,s')\Rightarrow\func{cut}\,(s) \wedge \func{cut}\,(s') \wedge (\func{exit}\,(s)\Leftrightarrow\func{exit}\,(s'))\\
2. & \func{pre}\,(s,s')\Rightarrow\func{C}\,(s,s') \wedge \func{assert}\,(s,s')\\
3. & \func{exit}\,(s) \Rightarrow \func{cut}\,(s) \\
4. & \func{C}\,(s,s')\wedge\func{assert}\,(s,s')\wedge\neg\func{exit}\,(s) \wedge\func{exit}\,(\func{run}\,(s,n))\\
& \Rightarrow\func{assert}\,(\func{nextc}\,(\func{next}\,(s)),\func{nextc}\,(\func{next}\,(s'))) \\
5. & \func{C}\,(s,s')\wedge\func{assert}\,(s,s') \wedge\neg\func{exit}\,(s) \wedge\func{exit}\,(\func{run}\,(s,n)) \\
& \Rightarrow\func{C}\,(\func{nextc}\,(\func{next}\,(s)),\func{nextc}\,(\func{next}\,(s'))) \\
6. & \func{assert}\,(s,s')\wedge\func{exit}\,(s)\wedge\func{C}\,(s,s')\Rightarrow\func{post}\,(s,s') 
\end{array}
\end{math}
\end{minipage}
}
\caption{Verification conditions for information flow.  Each
verification condition is implicitly quantified over all free
variables.  Here $\omega$ is the first infinite ordinal.  The function
$\func{cstps}\,(s)$ returns the number of transitions to the closest
cutpoint reachable from $s$ if one exists, and $\omega$ otherwise.
The reasons for returning $\omega$ when no cutpoint is reachable from
$s$ are technical, and not germane to this paper.}
\label{fig:vcg}
\end{figure}
Condition~$4$ formalizes the VCG guarantee.  We now discuss the
predicate \func{C}, which formalizes the notion of ``corresponding
cutpoints.''  Recall that the VCG guarantee specifies that when
\func{assert} holds between a pair of cutpoints then it also holds for
the subsequent pair.  However, the next subsequent cutpoints might be
out of sync.  For instance, computation might exit from $p$ and not
from $p'$; thus the information flow theorem cannot be derived from
the VCG guarantee.  The predicate \func{C} eliminates this possibility
by requiring the following: (1)~if $\func{pre}\,(s,s')$ holds then $s$
and $s'$ must be corresponding; (2)~for any two corresponding
cutpoints, the subsequent cutpoint pair must be corresponding; and
(3)~if $s$ satisfies \func{exit} and $s'$ is a corresponding cutpoint,
then $s'$ must also satisfy \func{exit}.  We assume that there is a
binary predicate \func{C} on $S\times S$ characterizing the
corresponding cutpoints; in practice, the definition of
$\func{C}\,(s,s')$ will usually reduce to the condition that the pc
values for $s$ and $s'$ are equal.

Conditions $4$ and $5$ involve multiple steps of program execution.
Contrary to common practice, we discharge them {\em without a VCG} as
follows.  We prove the following two theorems, which are easy
consequences of the definition of \func{nextc}:
\begin{description}
\item[SSR1:] 
$\neg\func{cut}(s)\Rightarrow\func{nextc}(s)=\func{nextc}(\func{next}(s))$
\item[SSR2:] $\func{cut}(s)\Rightarrow\func{nextc}(s)=s$
\end{description}
We treat SSR1 and SSR2 as oriented conditional equations or {\em
rewrite rules}.  For any symbolic state $s$, the rules rewrite the
term $\func{nextc}\,(s)$ to either $s$ or
$\func{nextc}\,(\func{next}\,(s))$; in the latter case the definition
of \func{next} is symbolically expanded, simplified, and the rules
applied again.  The proof attempt causes the theorem prover to
symbolically simulate the program from a cutpoint until the next
cutpoint is reached; the process mimics a forward VCG.  For symbolic
simulation to terminate, each program loop must contain a cutpoint, as
with traditional VCGs.

We briefly remark on how to automate the method above in ACL2.  The
derivation of noninterference from Conditions~$1$--$6$ is independent
of the definitions of \func{pre}, \func{post}, etc.  This allows the
development of a {\em proof template} for generating the verification
conditions as follows.  First, we introduce functions \func{pre},
\func{post}, \func{cut}, \func{assert}, \func{C}, and \func{next}
constrained to satisfy Conditions~$1$--$6$, and prove the
noninterference theorem for these constrained functions.\footnote{This
proof has been completed with ACL2 at the time of this writing.}  We
can then implement an ACL2 macro that automates the information flow
proofs as follows:
\begin{itemize}
\item Mechanically generate the concrete version of \func{nextc}.
\item Establish conditions~$1$--$6$ for the concrete versions of \func{pre},
\func{post}, etc., using symbolic
simulation.
\item Derive the information flow theorem by functionally
instantiating the generic version.
\end{itemize}
We have developed a corresponding proof template for (partial and
total) functional correctness~\cite{symbolic}.  We are working on
extending the template for information flow properties.  


\section{A ``Tricky'' Example}
\label{sec:tricky}

Our approach, although extremely simple, nevertheless provides a
scalable framework for information flow analysis.  One key strength is
the ability to use expressive predicates for proving information flow
theorems: if an information flow property depends on functional
invariants of the system state, then assertions can easily account for
such invariants.  This is in stark contrast to traditional security
type systems for information flow verification, which depend on the
syntactic analysis of the program to deduce information
flow~\cite{sm}.

As an illustration, consider the program shown in
Fig.~\ref{fig:tricky}.
\begin{figure}
\begin{center}
\fbox{
\begin{minipage}{3.0in}
\input{tricky1}
\end{minipage}
}
\end{center}
\caption{Pseudo-code for the Tricky Program.  Variable {\tt out} is
incremented by {\tt temp} only when {\tt i} is even, and {\tt temp} is
equal to {\tt low} in that case.  Thus, the final value of {\tt out}
is independent of {\tt high}.}
\label{fig:tricky}
\end{figure}
The information flow specification for the program is that the final
value of {\tt out} depends only on the initial values of {\tt low} and
{\tt n}.  The program is adapted from one in a recent paper by Amtoft
and Banerjee~\cite{tricky}\footnote{The difference between Amtoft and
Banerjee's program and that shown in Fig.~\ref{fig:tricky} is that the
former involved $7$ iterations of {\tt i} in the loop while we use
{\tt n} iterations.  Note that if the number of loop iterations is a
constant then the loop can be unrolled by symbolic simulation,
obviating loop invariants.}  which, although simple, was motivated by
an actual program used in operational verification of hardware
amplifiers provided by Rockwell Collins.  The information flow
property depends on a key observation: whenever the value of {\tt out}
is incremented by {\tt temp}, the value of {\tt i} is even and the
value of {\tt temp} is equal to {\tt low}; thus the final value of
{\tt out} is dependent only on {\tt low} (and the loop count {\tt n}).
The property cannot be inferred by type reasoning which would infer
dependence of {\tt temp} on {\tt high} and {\tt out} on {\tt temp}.

We formalized the program through an operational semantics of a simple
machine model, and proved the information flow specification using
inductive assertions.  The precondition stipulates that $s$ and $s'$
are poised to execute the program and the values of {\tt low}, {\tt
n}, and {\tt out} are the same in both states; the postcondition
specifies that the value of {\tt out} is the same after exiting the
program.  The only ``creative'' assertion is in the loop invariant.
In addition to the boiler-plate assertion that the values of {\tt
low}, {\tt n}, {\tt out}, and {\tt i} are the same in $s$ and $s'$, we
need the condition that if {\tt i} is even $s$ and $s'$ have the same
value of {\tt temp}.  With this assertion, the verification conditions
shown in Fig.~\ref{fig:vcg} are easily verified through symbolic
simulation.

It is instructive to compare our approach with that of Amtoft and
Banerjee~\cite{tricky}.  Their approach is built around the axiomatic
semantics for a special logical construct $\Join$ stipulating {\em
agreement assertions}: for a variable $x$, two states $p$ and $q$
satisfy $x\Join$ if and only if $x(p)=x(q)$.  They develop axiomatic
semantics for specifying loop flow and object flow using $\Join$, and
a VCG for the semantics.  In contrast, we generate and discharge the
verification conditions directly through symbolic simulation of the
operational semantics.  Nevertheless, our approach requires no more
creative insight than theirs, namely manually constructing the loop
invariant above.  On the other hand, our approach can harness the
power of a general-purpose theorem prover for symbolic simulation and
requires no axiomatic semantics for information flow.


\section{Compositionality}
\label{sec:compositionality}

The above treatment did not consider compositionality.  Consider
verifying a program \proc{P} that invokes a subroutine
\proc{Q}. Symbolic simulation from a cutpoint of \proc{P} might
encounter an invocation of \proc{Q}, resulting in simulation of
\proc{Q}.  We prefer to separately verify \proc{Q}, and use the
result for verifying \proc{P}.

To achieve composition, we must handle the {\em frame conditions} to
justify that \proc{P} can continue execution after \proc{Q} returns.
Note that an information flow property of \proc{Q} is not sufficient
for this; for instance, we must show that the execution of \proc{Q}
does not corrupt the call stack.  For functional correctness, the
frame problem is addressed by phrasing the postcondition as an
equality to characterize how each state component is modified by
\proc{Q}~\cite{symbolic}.  However, a full characterization of
\proc{Q} is often irrelevant to the information flow of \proc{P}.  The
challenge is to effectively augment the postcondition of \proc{Q} with
frame conditions to facilitate symbolic simulation of \proc{P}.

How do we address the challenge?  Let $V$ be the set of state
components governing the control flow on return from \proc{Q};
typically $V$ includes the call stack.  Then we define
$\func{modify}_Q\,(s)$ to update each component of $s$ as follows.
For each component $v\in V$, $\func{modify}_Q\,(s)$ updates $v$ by
precisely characterizing its modification on exit from \proc{Q}. The
update to the call stack is characterized by popping the current call
frame.  For $c\not\in V$, the update is simply
$c(\func{nexte}_Q\,(s))$, which may or may not need to be
characterized depending on the caller.  We then prove the following
two conditions.

\begin{enumerate}
\item 
$\func{poise}_Q\,(s) \wedge \func{exit}_Q\,(\func{run}\,(s,n))
\Rightarrow\func{nexte}_Q\,(s)=\func{modify}_Q\,(s)$
\item
$\func{pre}_Q\,(s,s') \wedge \func{exit}_Q\,(\func{run}\,(s,n))
  \Rightarrow \\
\func{post}_Q\,(\func{modify}_Q\,(s),\func{modify}_Q\,(s'))$ 
\end{enumerate}

\noindent
Here, $\func{poise}_Q\,(s)$ states that $s$ is poised to invoke
\proc{Q}. Its definition is derived from $\func{pre}_Q\,(s,s')$ by
collecting the conjuncts that only mention $s$.  Condition~2 follows
from 1 and the information flow property of \proc{Q}.  We prove 1
through inductive assertions, viewing $\func{modify}_Q$ as a functional
characterization of \proc{Q}~\cite{symbolic}.  The necessary
assertions can be culled from the information flow proof of \proc{Q}.
Recall that the predicate $\func{assert}_Q\,(s,s')$ is strong enough to
characterize the flow of control from each cutpoint of $s$ (and $s'$)
to the next. Thus, the conjuncts in $\func{assert}_Q\,(s,s')$ that only
involve $s$ can be used for symbolic simulation from each cutpoint in
$s$ to prove 1.  The theorems facilitate compositional reasoning.  If
states $s$ and $s'$ encountered during symbolic simulation of \proc{P}
are poised to execute \proc{Q}, then 1 permits simulation to ``skip
past'' \proc{Q}, and 2 enables us to assume \func{post} on the
generated pair $(\func{modify}_Q\,(s), \func{modify}_Q\,(s'))$ during subsequent
simulation.  Note that this can be automated using macros as hinted at
in Section \ref{sec:basic-framework} for the basic framework.

We used the approach above to compositionally verify the program
shown in Fig.~\ref{fig:main}.
\begin{figure}
\begin{center}
\fbox{
\begin{minipage}{3.0in}
\input{mainprog}
\end{minipage}
}
\end{center}
\caption{A Program to demonstrate compositionality}
\label{fig:main}
\end{figure}
The program is artificial but illustrative.  It invokes one of two
separate versions of the {\tt tricky} procedure depending on {\tt
flag}: {\tt tricky1} is as shown in Fig.~\ref{fig:tricky}; {\tt
tricky3} iterates {\tt 3n} times instead of {\tt n}.  Our information
flow specification is that the final value of {\tt out} on \func{exit}
from {\tt main} is independent of {\tt high}.  Note that the
information flow analysis of {\tt tricky3} is exactly analogous to
{\tt tricky1}, but its return value is different.  Thus, a complete
functional characterization of the two routines would involve separate
analysis.  However, the actual return value of the subroutines is
immaterial to the information flow of {\tt main}.  With our approach,
{\tt main} can be verified using only the noninterference of each
subroutine and the frame conditions.


\section{Related Work and Conclusion}
\label{sec:related}

Information flow analysis was formulated by Denning and
Denning~\cite{denning}.  Sabelfeld and Meyers~\cite{sm} contains a
comprehensive survey of the area.  Traditional approaches to
information flow analysis involves {\em security type
systems}~\cite{palsberg,volpano}: program variables and expressions
are annotated with security levels, and flow of information is controlled
by typing rules.  There has also been significant recent work on
axiomatic semantics for information flow.  Clark {\em et
al.}~\cite{clark} develop a semantics for Idealized Algol.  Joshi and
Leino~\cite{joshi} develop a weakest precondition calculus for
information flow.  Darvas {\em et al.}~\cite{darvas} use dynamic logic
to express information flow for Javacard.

Our work provides the first framework for information flow analysis
through inductive assertions directly on operational semantics.  No
separate VCG or axiomatic semantics for information flow is necessary.
Instead, the generation and discharge of verification conditions are
handled by the theorem prover through symbolic simulation.
Furthermore, we can compose information flow properties of subroutines
without requiring full characterization of their functional
specification.  The framework is in an early stage of development.
As mentioned in Section~\ref{sec:basic-framework}, we are developing
proof templates to facilitate the automation of information flow
verification.  Some planned future enhancements include (1)~automated
static analysis of data structure shapes, (2)~extension to
multithreaded programs, and (3)~analysis of dynamic and
declassification policies.

\vspace*{2mm}

\subsubsection*{Acknowledgements}
This material is based upon work supported by DARPA and the National
Science Foundation under Grant No.~CNS-0429591. Matt Kaufmann provided
numerous comments and suggestions in course of this work.

\bibliographystyle{IEEEtran} 
\bibliography{information-flow}
\end{document}


---------------------------------------------

Paper: 43
Title: Mechanized Information Flow Analysis through Inductive Assertions


-------------------- review 1 --------------------

PAPER: 43
TITLE: Mechanized Information Flow Analysis through Inductive Assertions
 
OVERALL RATING: 1 (weak accept)   
----------------------- REVIEW --------------------

This paper presents an approach to information flow analysis (as
in security) in ACL2.  The authors' "inductive assertions"
method appears very standard:  annotate the with inductive
assertions, and then discharge Hoare-style verification conditions
automatically between the inductive assertions.  In practice,
coming up with the inductive assertions is always the problem,
but for a high-value target like security, such an expensive
approach may be warranted.

This is solid work, and the paper is very well-written.
And it is apparently much better than the approach proposed
by Amtoft and Banerjee that the authors cite.  However, it
doesn't seem particularly novel:  annotating a program with
inductive assertions and using a theorem-prover to discharge
the resulting verification conditions has been standard for
30 years.  The paper notes "A key feature of our approach is
that it obviates the need for implementing a custom verification
condition generator (VCG)," but the proposed method is a
customized VCG.  It's just a clean way to implement one. 


-------------------- review 2 --------------------

PAPER: 43
TITLE: Mechanized Information Flow Analysis through Inductive Assertions
 
OVERALL RATING: -2 (reject)   
----------------------- REVIEW --------------------

This paper shows how to use inductive assertions and a mechanized theorem prover to establish information-flow properties (such as non-interference) of software programs.

The work is an extension of [2].

It is not clear to me how much is new in this paper as opposed to [2], other than the application area. Applying safety property verification methods to non-interference has been done before, even in the context of automated theorem proving, see for example:

"A Theorem Proving Approach to Analysis of Secure Information Flow" by Adam Darvas, Reiner Hahnle, David Sands, WITS'03.

It is not clear to me what exactly is manual and what is automatic in the method. The loop invariant has to be asserted manually, what is automatic?

About the compositionality: One big advantage with a type-system approach is that it is (by definition) compositional, and therefore predictable. You claim that your method is compositional, but it is not really. For example, if I lifted out the body of the for-loop in Fig. 3 and put it in a procedure, and used the procedure instead in the for-loop, you would have to do functional reasoning about that procedure in order to get the same result.

How does your method deal with information leakage through non-termination? 


-------------------- review 3 --------------------

PAPER: 43
TITLE: Mechanized Information Flow Analysis through Inductive Assertions
 
OVERALL RATING: 1 (weak accept)   
----------------------- REVIEW --------------------

The paper presents a tool for checking information flow properties
through theorem-prover ACL2. As opposed to checking Hoare triples of
programs, the formulation of the information flow given in the paper
requires checking assertions over pairs of states. Hence, conventional
verification condition generation tools like ESC/JAVA can't be used
directly for this purpose. The paper presents a VC generator for
programs annotated with information flow assertions at procecdure and
loop boundaries. 

The basic idea is to define a set of predicates over pairs of states
(a) assert(s,s') that is a set of assertions over s, s', (b) C(s,s')
that denote if two states are in "corresponding" cutpoints, (c)
pre(s,s') and post(s,s') for the pre and post relating the information
flow property. The VCG guarantees that two states s, s' at cutpoints
satisfying C(s,s') and satsify the assertion assert(s,s') and they
both transition to another pair of corresponding cutpoints whose
assertions are satisfied. Moreover, for corresponding cutpoints
corresponding to exit locations, the assertions satisfy the post
conditions. 

Strengths and weaknesses:

+ The paper describes a new technique for checking information flow
properties with a theorem prover. The approach has the promise of
being more semantic and precise compared to type checkers. 

+ A simple but illustrative example has been worked that demonstrates
the need for expressive assertions to prove the information flow
property along with necessary assertions. 

- The approach requires the user to annotate every pair of cutpoint in
  the program with assertions that makes it much more complex compared
  to checking Hoare triples or functional correctness. The author does
  not shed any light in this direction. Moreover, the user might
  sometime need to intervene with the theorem prover to prove complex
  conjectures. 

- The paper does not show any example of how C(s,s') might look
  like in general. It is trivial for the example in Fig 3 since two
  corresponding cutpoints have the same pc. 

- No implementation has been done at this stage. 

- The description of the compositionality in Section IV is very hard
  to follow, especially how to use postconditions (apart from the
  modifies clauses) of a called  procedure. It is unclear to me why we
  have to ensure that the call stack is not corrupted on return from a
  procedure. 



Question:

1. Given an annotated progaram with information flow assertions at
   cutpoints, what is the complexity of the resulting VC formula? Can
   its size be limited by some polynomial bound over the input program
   like the VC generation using weakest precondition calculus?


2. What is the specification (e.g. modify_Q) of the two procedures
   tricky1 and tricky3 in the example of Fig 4? 


-------------------- review 4 --------------------

PAPER: 43
TITLE: Mechanized Information Flow Analysis through Inductive Assertions
 
OVERALL RATING: 1 (weak accept)   
----------------------- REVIEW --------------------

The score reflects the combination of three reviews, two of which are in favor of the paper's inclusion and one mildly in favor of its rejection.  Overall, these reviewers thought this was promising work, and we all found some detail to be lacking (yes, I know it's only a a short paper).  If the authors provided their implementation on-line, some uncertainties could have been shored up.  Additional shortcomings include

  * A lack of detailed comparisons with related work in non-interference analysis, particularly using a theorem-prover;
  * A mis-directed emphasis on obviating the need for a VCG.  That is not what these reviewers found to be novel.
  * Being unconvinced about the compositionality of the technique.
  * These reviewers would like to see more emphasis on inductive assertions for information flow properties.  

Strengths include:
  * Being well-written.
  * Being highly-relevant to FMCAD.
  * Tackling an interesting problem using seemingly novel techniques, and actually implementing the framework.

Overall, I am assigning the paper a "weak accept."


**PC member's review**

This work appears to be very much derived from reference [2].  (As an aside, I find it slightly strange to see the sentence "The method is inspired by, and an extension of, our previous work [2] which showed how to prove functional correctness via symbolic simulation."  The only intersection of authors is that the 3rd author (out of four) of the present paper is the 3rd author (out of four) of the cited paper.)  I am somewhat familiar with [2].  One could argue that the only research contribution of this paper is to demonstrate that the "state-based VCG" technique presented in [2] to prove functional correctness can also be used to prove non-interference properties.  That is not an extraordinary result; it is somewhat obvious.  However, it is nice to see that it works in practice, in ACL2, at least for a very small contrived example.

The authors themselves note that the framework is in the early stages of development, and this is a short paper.  If this paper were a full paper, I could imagine it being taken in two directions. (1) Spend more time comparing the framework here with other approaches (e.g., axiomatic) to prove noninterference, so the reader has a good sense of what is the best approach, in a theorem-prover.  (2) The other direction would be to extend the kinds of use-cases and framework, and focus on how the approach works in practice on larger programs.

I would have liked to have seen more comparisons with recent information flow approaches.  For example, in "Verified Safety and Information Flow of a Block Device" by Paul Graunke in Systems Software Verification (SSV 08), a monadic approach is used to prove noninterference in a theorem-prover (it was done in Isabelle, and the system was specified monadically).  In general, I was surprised about the lack of related work cited for verifying noninterference in mechanical theorem-provers.

Also, I'm not so interested in the emphasis (over and over) about there being no VCG.  I feel like the community already knows that there many ways to obviate the need for a VCG or how to build one systematically.  This is an unfounded opinion, but the focus on obviating the VCG almost makes the paper sound dated.

I'm on the fence.  The paper is well-written, in my opinion, the result is clear, and the example nicely demonstrates the approach.  I would encourage the authors to strengthen this paper and to submit it later.  It will make a greater impact if strengthened, but publishing this paper could reduce the chances of a longer paper being accepted.


**Sub-reviewer #1**
Summary: 

This paper describes an approach for verifying information flow
properties in software systems. A program is annotated with assertions at
appropriate "cut points" describing restrictions on information flow. This
collection of annotations along with an operational semantics are then used to
verify flow properties using ACL2. The authors claim that their verification
steps are compositional, requiring that each subroutine be analyzed once rather
than at each call site. This is a work in progress with the implementation in
ACL2 ongoing.

The operational semantics used to specify program behavior is a routine
application of state-based semantics, seemingly axiomatic. As an example of
verification conditions, noninterference is defined. Cut points appear to be
identified as loop conditions and subprogram entry and exit points and define
where annotations must be added. (Thus I think supporting the argument for
compositionality.) The authors define nextc as a mechanism for finding the next
cut point with respect to a given cut point and define verification conditions
for information flow. The approach seems like a quite straightforward
application of axiomatic semantics. If not, I would like to know more details
concerning why.

For compositionality, the authors discuss the use of frame conditions that
describe specific conditions that must be preserved across a subroutine call.
They provide a reasonable description of how frame conditions are defined and a
simple example. Again, using axiomatic semantics in this manner is well
established. How does this extend what has already been done?

Evaluation:

The authors say themselves that the approach "appears extremely simple". I
definitely agree, but simple is not a bad thing. Quite the contrary! Having said
that, I would like to see evidence of the approach being applied to programs
longer than a 15-line C program. Particularly with respect to composability
arguments. It's also not completely clear to me if this can address concurrent
programs where access control presents much more difficult problems. In their
defense, the authors do clearly state that this is a work in progress.
Definitely worth hearing more about at the conference.

Could you say more about what you mean by, "The property cannot be inferred by traditional type reasoning."?

Points in Favor:
+ Well written and reasonably easy to follow.
+ Describes an approach suitable for automation and shows promise for scaling up
  to large systems.
+ Composability is always a good result and useful to hear about.

Points Against:
- In section III, the authors argue the approach is scalable with no
  evaluation to support the claim.  Lots of things look scalable that
  are not.
- Annotating cut points in large programs is time consuming and may prove
  difficult. Do you have a way around this?
- Defining a full operational semantics for most software systems will
  definitely prove difficult and result in enormous mathematical theories. Do
  you have an approach for this?

Typo:

"information flow control;" --> "information flow control,"


**Sub-reviewer #2**
The paper describes a method for proving noninterference properties of
programs, using Hoare-style annotations at cutpoints where the pre
(and post) conditions are relations over pairs of start (and end)
states.

In common with previous work by the authors, the symbolic simulation
technique obviates the need for implementing a custom verification
condition generator. However, the more significant contribution of the
paper is the relation form of pre and postconditions, which: (i)
supports compositional reasoning over programs; (ii) allow conditional
noninterference properties to be expressed; and (iii) require no new
logical constructs (e.g., Amtoft and Banerjee's bowtie connective). If
novel, this by itself is worthy of publication, and the fact that the
analysis is mechanized in a well known theorem prover makes this a
solid piece of work. 

